\documentclass[a4paper,11pt]{article}
\usepackage[T1]{fontenc}
\usepackage[utf8]{inputenc}
\usepackage{lmodern}

\title{B-Tag 2015}
\author{Thomas, Josua, Niclas, Andreas}

\begin{document}

\maketitle
\tableofcontents

\section{Aufgaben}
\subsection{Aufgabe 1: Dreiecksgeometrie}
Wir wollen eine Funktion $\overline{FE}(\theta)$ aufstellen, und zeigen, dass diese immer gr\"o\ss er als CA ist.

1. Wie lang ist die Strecke $\overline{FM}$?
\[ \overline{FM}(\theta) = \frac{M_y}{sin(\theta)} \]

2. Wie lang ist die Strecke $\overline{ME}$?
\[ \overline{ME}(\theta) = \frac{M_x}{sin(90-\theta)} \]

3. Die Strecke $\overline{FE}$ ist also $\overline{FE} + \overline{ME}$:
\[ \overline{FE}(\theta) = \frac{M_y}{sin(\theta)} + \frac{M_x}{sin(90-\theta)} \]

\end{document}
