\documentclass[a4paper,11pt]{article}
\usepackage[T1]{fontenc}
\usepackage[utf8]{inputenc}
\usepackage{lmodern}
\usepackage{graphicx}
\usepackage[german]{babel}

\title{B-Tag 2015}
\author{Thomas, Josua, Niclas, Andreas}

\begin{document}

\maketitle
\tableofcontents

\section{Aufgaben}
\subsection{Aufgabe 1: Dreiecksgeometrie}
\textbf{Ist $\overline{FE}$ länger als $\overline{CA}$?}

\begin{figure}[htbp] 
        \centering
        \includegraphics[width=6cm]{img/A1_1.png}
\end{figure}

Wir wollen eine Funktion $\overline{FE}(\theta)$ aufstellen, und zeigen, dass diese immer gr\"o\ss er als $\overline{CA}$ ist.

1. Wie lang ist die Strecke $\overline{FM}$?
\[ \overline{FM}(\theta) = \frac{M_y}{sin(\theta)} \]

2. Wie lang ist die Strecke $\overline{ME}$?
\[ \overline{ME}(\theta) = \frac{M_x}{sin((\pi/2)-\theta)} \]

3. Die Strecke $\overline{FE}$ ist also $\overline{FM} + \overline{ME}$ (natürlich alles im Definitionsbereich $0 < \theta < \frac{\pi}{2}$:
\[ \overline{FE}(\theta) = \frac{M_y}{\sin(\theta)} + \frac{M_x}{\sin((\pi/2)-\theta)} \]

Jetzt muss gezeigt werden, dass der Tiefpunkt von $\overline{FE}(\theta)$ den Wert $\overline{AC}$ hat. Dazu wird $\overline{FE}(\theta)$ zuerst abgeleitet, um den TP zu finden:

%\[ \overline{FE}'(\theta) = - \frac{M_y \cos(\theta)}{(\sin(\theta))^2} + \frac{M_x \cos(\theta - (\pi/2))}{\sin((\pi/2)-\theta)^2} \]
\[ \overline{FE}'(\theta) = \frac{M_x (\sin(\theta))^3 - M_y (\cos(\theta))^3)}{(\sin(\theta))^2 (\cos(\theta))^2} \]

Jetzt setzen wir $\overline{FE}' = 0$, um den Tiefpunkt von $\overline{FE}(\theta)$ bei $\theta = \frac{\pi}{4}$ zu finden, und sehen, dass $FE(\frac{\pi}{4}) = CA$ ist. Daher ist $\overline{FE}$ immer länger als $\overline{CA}$ (au\ss er bei $\theta=\frac{\pi}{2}$).

\subsection{Aufgabe 2: Verschiebungen}

\begin{figure}[htbp] 
        \centering
        \includegraphics[width=8cm]{img/A2_1.png}
\end{figure}

\textbf{Wo ist das Drehzentrum R?}
Die Aufgabenstellung verlangte die Bestimmung eines Drehzentrums anhand 2 identischer, jedoch in der Lage 
veränderten Figuren, anhand zwei sich auf den Figuren befindenden Punkten. Zuerst machten wir uns folgende Bedingungen zu Nutze:

\[ |AR| = |A'R| \]
\[ |BR| = |B'R| \]

Wenn man nun um die verschobenen Punkte A und A' jeweils einen Kreis mit gleichem Radius legt, welcher so gewählt wird, sodass sich die Kreise schneiden.\\
Entlang der beiden Schnittpunkte lässt sich eine Gerade legen. Ebenso verfährt man mit Punkt B bzw. B'. \\
Das Drehzentrum R befindet sich im Schnittpunkt der beiden Geraden.

\subsection{Aufgabe 3: St\"ocke}

\begin{figure}[htbp] 
        \centering
        \includegraphics[width=6cm]{img/A3_1.png}
\end{figure}

\textbf{Welche St\"ocke passen auf jeden Fall nicht durch?} (Angenommen, der Gang ist 1 breit) \\
Die Stelle, an der die maximale Länge am kleinsten ist, ist wenn der Stock in einem $\frac{\pi}{2}$ Winkel zu den Wänden steht. Deshalb darf ein Stock auf jeden Fall nicht länger als die Diagonale an dieser Stelle sein, also $2\sqrt{2}$.

\textbf{Passt der $2\sqrt{2}$ Stock durch den Gang?} \\
Wir nehmen an, der Stock berührt die ganze zeit die innere Ecke (damit er durch passt). Dann haben wir die Gleiche Situation wie in A1. Die maximale Länge des Stocks wird immer größer, je weiter die Winkel zu den Wänden von $\frac{\pi}{2}$ sind. Deshalb passt ein Stock bei jedem Winkel, wenn er auch bei $\frac{\pi}{2}$ passt.

\subsection{Aufgabe 4: Strategien f\"ur den l\"angsten Stock}

\subsection{Aufgabe 5: Gr\"o\ss te Rechtecke}

\textbf{Wie groß ist der Flächeninhalt des größten Rechtecks, das um die Ecke bewegt werden kann,
wenn die kürzere Seite des Rechtecks die Länge 0,5 hat?} \\
Wenn die H\"ohe des Rechtecks 0.5 ist, ist die gr\"o\ss te Breite 1. Dann hat man eine Fl\"ache von 0.5.
Wenn die Breite 0.5 ist, muss erst einmal ausgerechnet werden, was die maximale H\"ohe ist. Das Rechteck muss im Winkel von $\frac{\pi}{2}$ noch passen: 

\begin{figure}[htbp] 
        \centering
        \includegraphics[width=6cm]{img/A5_1.png}
\end{figure}

\[ b = \sqrt{2} - 0.5 \]
\[ \tan(\frac{\pi}{2}) = \frac{2b}{a} \Leftrightarrow  a = \frac{2b}{\tan(\pi/2)} \]

$a$ w\"are also in dem Fall ca. 1.83, was einen Fl\"acheninhalt von ca. 0.914 ergibt.


\end{document}
